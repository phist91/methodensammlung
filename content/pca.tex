%!TEX root = ../methoden.tex
% Author: Phil Steinhorst, p.st@wwu.de
\section{Hauptkomponentenanalyse (PCA)}
\label{sec:pca}
\begin{keywordbox}
	\keyword{Hauptkomponentenanalyse}, \keyword{Faktorenanalyse}, \keyword{principal component analysis}, \keyword{PCA}, \keyword{explorativ}, \keyword{konfirmatorisch}, \keyword{Reduktion}, KMO, Kaiser-Meyer-Olkin, Bartlett, Sphärizität, Kaiser-Kriterium, Eigenwert, erklärte Varianz, Scree Plot
\end{keywordbox}
\laerdlink{https://statistics.laerd.com/premium/spss/pca/pca-in-spss.php}{PCA in SPSS}

\subsection{Anwendungsfälle}
\begin{itemize}
	\item Reduziert gemessene Variablen auf eine kleinere Menge an \textit{künstlichen} Variablen, die für die meiste Varianz in den Messwerten verantwortlich sind
	\item Konzeptionell unterschiedlich zur explorativen Faktorenanalyse, aber häufig gleiches Verfahren
	\item Variablen werden geclustert, um evtl. irrelevante Variablen herauszufiltern
	\item Zusammenfassung von Variablen, die das gleiche Konstrukt messen, zu einem \textit{component score}
\end{itemize}

\subsection{Voraussetzungen}
\begin{enumerate}[(1)]
	\item Mehrere Variablen mit stetiger oder ordinaler Skala
	\begin{alertbox}
		Auf umgekehrt kodierte Variablen achten!
	\end{alertbox}
	\item Lineare Beziehung zwischen den Variablen
	\item Keine Ausreißer
	\item Große Stichprobe: mindestens 150 Datensätze bzw. 5--10 Datensätze pro Variable
\end{enumerate}
Voraussetzungen (2) bis (4) werden während des Verfahren überprüft.

\subsection{Durchführung in SPSS}
\begin{enumerate}
	\item Daten importieren, Variablen deklarieren
	\item Menü \textit{Analyze} $\rightarrow$ \textit{Dimension Reduction} $\rightarrow$ \textit{Factor\dots}
	\item Zu faktorisierende Variablen auswählen, Einstellungen vornehmen:
	\begin{description}
		\item[Descriptives:] Univariate Descriptives, Initial Solution, Coefficients, KMO and Bartlett, Reproduced, Anti-Image
		\item[Extraction:] Methode auswählen, Correlation Matrix, Unrotated factor solution, Scree plot, Eigenvalues greater than 1
		\begin{description}
			\item[Principal components:] 
			\item[Principal axis factoring:] 
		\end{description}
		\item[Rotation:] Methode auswählen, Rotated solution, Loading plot(s)
		\begin{description}
			\item[None:] 
			\item[Varimax:]
			\item[Direct Oblimin:] Delta 0
		\end{description}
		\item[Options:] Missing Values auswählen, Sorted by size, Suppress small coefficients below 0.3
		\begin{description}
			\item[Listwise exclusion:] Datensätze mit fehlenden Werten überall außen vor lassen (empfohlen)
			\item[Pairwise exclusion:] Verwendet alle möglichen Datensätze (Vorsicht: Unterschiedliche Stichproben!)
			\item[Replace with mean:] Selbsterklärend, aber meistens Quatsch
		\end{description}
	\end{description}
\end{enumerate}

\subsection{Voraussetzungen prüfen}
\begin{enumerate}
	\item \textit{Correlation Matrix:} Gibt es eine Variable, die nicht stark genug mit einer anderen korreliert?
		\begin{alertbox}
			Variablen mit $r < 0.3$ aus Analyse fortlassen und wiederholen, Variablen mit $r < 0.4$ im Auge behalten!
		\end{alertbox}
	\item \textit{Kaiser-Meyer-Olkin-Maße:} Misst Eignung der Stichprobe (Linearität) \index{Kaiser-Meyer-Olkin}\index{KMO}
		\begin{description}
			\item[Measure of Sampling Adequacy:] mindestens $0.6$, besser $0.8$
			\item[Anti-Image Correlation:] mindestens $0.5$ auf Diagonale, besser $0.8$
		\end{description}
		\begin{alertbox}
			Variablen mit $r < 0.5$ in der Anti-Image Correlation Matrix aus Analyse entfernen!
			Bei Sampling Adequacy $< 0.6$ ist das Verfahren ungeeignet bzw. die Ergebnisse nicht aussagekräftig!
		\end{alertbox}
	\item \textit{Bartlett's Test of Sphericity:} Test gegen Nullhypothese: Korrelationsmatrix ist Einheitsmatrix (= keine Korrelation zwischen einzelnen Variablen $\Rightarrow$ keine Faktorisierung möglich).
	Statistische Signifikanz ($p < 0.05$) wünschenswert. \index{Bartlett's Test of Sphericity} \index{Sphärizität}
		\begin{alertbox}
			Bei $p > 0.05$ ist das Verfahren ungeeignet!
			Jedoch: Verfahren ist empfindlich gegenüber Abweichungen von der mehrdimensionalen Normalverteilung.
		\end{alertbox}
\end{enumerate}

\subsection{Kriterien zur Faktorenauswahl}
\begin{itemize}
	\item \textit{Kaiser-Kriterium:} Variablen mit Eigenwert $\geq 1$ \index{Kaiser-Kriterium}\index{Eigenwert}
	\item \textit{Erklärte Varianz:} individuell mindestens $5$--$10\%$, kumuliert mindestens $60$--$70\%$.
	\item \textit{Wendepunkt im Scree Plot} \index{Scree Plot}
	\item \textit{Simple Structure} in rotierter Komponentenmatrix:
		Jede Variable mit starker Auswirkung auf nur eine Komponente; jede Komponente mit starker Auswirkung auf mindestens drei Variablen.
\end{itemize}
Nach Festlegung der Faktoren: Rerun mit fester Faktorenzahl; Güte überprüfen

\todo[inline]{Gütekriterien der konfirmatorischen Faktorenanalyse: CFI, Tucker-Lewis, RMSEA, SRMR, GFI/AGFI}
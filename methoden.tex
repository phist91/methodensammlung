%!TeX encoding = UTF-8
%!TEX TS-program = pdflatex
% Author: Phil Steinhorst, p.st@wwu.de
%!TEX root = methoden.tex
% Author: Phil Steinhorst, p.st@wwu.de
\documentclass[%
	paper=a5,
	twoside,
	fontsize=9pt,
	BCOR=8mm,
	DIV=calc,
	bibliography=totoc
]{scrreprt}


% Kodierung und Sprache
%%%%%%%%%%%%%%%%%%%%%%%%%%%%%%%%%%%%%%%%%%%%%%%%%%%%%%%%%%%%%%%%%%%%%%
\usepackage[utf8]{inputenc}
\usepackage[T1]{fontenc}
\usepackage[ngerman]{babel}
\usepackage[german=quotes]{csquotes}
%%%%%%%%%%%%%%%%%%%%%%%%%%%%%%%%%%%%%%%%%%%%%%%%%%%%%%%%%%%%%%%%%%%%%%


% Fonts
%%%%%%%%%%%%%%%%%%%%%%%%%%%%%%%%%%%%%%%%%%%%%%%%%%%%%%%%%%%%%%%%%%%%%%
\usepackage{tgheros}
\usepackage{heuristica}
\usepackage[heuristica,vvarbb,bigdelims]{newtxmath}
\usepackage{nimbusmononarrow}
%%%%%%%%%%%%%%%%%%%%%%%%%%%%%%%%%%%%%%%%%%%%%%%%%%%%%%%%%%%%%%%%%%%%%%


% Layout
%%%%%%%%%%%%%%%%%%%%%%%%%%%%%%%%%%%%%%%%%%%%%%%%%%%%%%%%%%%%%%%%%%%%%%
\usepackage{setspace}
\setstretch{1.15}
\recalctypearea
\usepackage[babel=true,final,tracking=smallcaps]{microtype}
\DisableLigatures{encoding = T1, family = tt* }
\usepackage{ellipsis}
\raggedbottom
\setlength{\parindent}{0cm}
\setlength{\parskip}{0.5\baselineskip}
%%%%%%%%%%%%%%%%%%%%%%%%%%%%%%%%%%%%%%%%%%%%%%%%%%%%%%%%%%%%%%%%%%%%%%


% Kopf- und Fußzeile
%%%%%%%%%%%%%%%%%%%%%%%%%%%%%%%%%%%%%%%%%%%%%%%%%%%%%%%%%%%%%%%%%%%%%%
\usepackage[autooneside=false,automark]{scrlayer-scrpage}
\clearpairofpagestyles
\setkomafont{pageheadfoot}{\sffamily\footnotesize}
\setkomafont{pagehead}{\bfseries}
\setkomafont{pagination}{}
\KOMAoptions{
   headsepline = true,
   footsepline = true,
   plainfootsepline = true,
}

\ohead{\headmark}
\ofoot{\pagemark}
%%%%%%%%%%%%%%%%%%%%%%%%%%%%%%%%%%%%%%%%%%%%%%%%%%%%%%%%%%%%%%%%%%%%%%


% Farben
%%%%%%%%%%%%%%%%%%%%%%%%%%%%%%%%%%%%%%%%%%%%%%%%%%%%%%%%%%%%%%%%%%%%%%
\usepackage[usenames,x11names,final,table]{xcolor}
\definecolor{fbblau}{HTML}{3078AB}
\definecolor{mediumgray}{gray}{.65}
\definecolor{blackberry}{rgb}{0.53, 0.0, 0.25}
%%%%%%%%%%%%%%%%%%%%%%%%%%%%%%%%%%%%%%%%%%%%%%%%%%%%%%%%%%%%%%%%%%%%%%


% TikZ
%%%%%%%%%%%%%%%%%%%%%%%%%%%%%%%%%%%%%%%%%%%%%%%%%%%%%%%%%%%%%%%%%%%%%%
\usepackage{tikz}
\usetikzlibrary{arrows.meta}			% mehr Pfeile!
\usetikzlibrary{shadows}
\usetikzlibrary{calc}
\tikzset{>=Latex}						% Standard-Pfeilspitze
\usepackage[tikz]{mdframed}
%%%%%%%%%%%%%%%%%%%%%%%%%%%%%%%%%%%%%%%%%%%%%%%%%%%%%%%%%%%%%%%%%%%%%%


% Hyperref
%%%%%%%%%%%%%%%%%%%%%%%%%%%%%%%%%%%%%%%%%%%%%%%%%%%%%%%%%%%%%%%%%%%%%%
\PassOptionsToPackage{hyphens}{url}
\usepackage[%
	hidelinks,
	pdfpagelabels,
	bookmarksopen=true,
	bookmarksnumbered=true,
	linkcolor=black,
	urlcolor=fbblau,
	plainpages=false,
	pagebackref,
	citecolor=black,
	hypertexnames=true,
	pdfauthor={Phil Steinhorst},
	pdfborderstyle={/S/U},
	linkbordercolor=SkyBlue2,
	colorlinks=true,
	backref=false]{hyperref}
\renewcommand{\UrlFont}{\ttfamily}
\urlstyle{tt}
\hypersetup{final}
%%%%%%%%%%%%%%%%%%%%%%%%%%%%%%%%%%%%%%%%%%%%%%%%%%%%%%%%%%%%%%%%%%%%%%


% Indexerstellung
%%%%%%%%%%%%%%%%%%%%%%%%%%%%%%%%%%%%%%%%%%%%%%%%%%%%%%%%%%%%%%%%%%%%%%
\usepackage{makeidx}

\makeindex
%%%%%%%%%%%%%%%%%%%%%%%%%%%%%%%%%%%%%%%%%%%%%%%%%%%%%%%%%%%%%%%%%%%%%%


% Listen und Tabellen
%%%%%%%%%%%%%%%%%%%%%%%%%%%%%%%%%%%%%%%%%%%%%%%%%%%%%%%%%%%%%%%%%%%%%%
\usepackage{multicol}
\usepackage{multirow}
\usepackage[shortlabels]{enumitem}
\setlist{itemsep=0pt}
\setlist[enumerate]{font=\sffamily\bfseries}
\setlist[itemize]{label=$\triangleright$,font=\bfseries}
\usepackage{tabularx}
\usepackage[fulladjust]{marginnote}
\renewcommand*{\marginfont}{\itshape\footnotesize}
\usepackage[textsize=small,color=Red1!80!OrangeRed1!80]{todonotes}
%%%%%%%%%%%%%%%%%%%%%%%%%%%%%%%%%%%%%%%%%%%%%%%%%%%%%%%%%%%%%%%%%%%%%%


% Inhaltsverzeichnis und Querverweise
%%%%%%%%%%%%%%%%%%%%%%%%%%%%%%%%%%%%%%%%%%%%%%%%%%%%%%%%%%%%%%%%%%%%%%
\usepackage[tocindentauto]{tocstyle}
\usetocstyle{KOMAlike}	
\usepackage[nameinlink]{cleveref}
\usepackage{chngcntr}
\usepackage[font=footnotesize]{caption}
%%%%%%%%%%%%%%%%%%%%%%%%%%%%%%%%%%%%%%%%%%%%%%%%%%%%%%%%%%%%%%%%%%%%%%


% Bibliografie
%%%%%%%%%%%%%%%%%%%%%%%%%%%%%%%%%%%%%%%%%%%%%%%%%%%%%%%%%%%%%%%%%%%%%%
\usepackage[natbib,style=alphabetic,minnames=2,maxnames=2,uniquelist=false,backend=biber]{biblatex}
\bibliography{bib}
\setlength{\bibitemsep}{1.5em}  
%%%%%%%%%%%%%%%%%%%%%%%%%%%%%%%%%%%%%%%%%%%%%%%%%%%%%%%%%%%%%%%%%%%%%%


% Debugging
%%%%%%%%%%%%%%%%%%%%%%%%%%%%%%%%%%%%%%%%%%%%%%%%%%%%%%%%%%%%%%%%%%%%%%
%\usepackage{showframe}
\usepackage{lipsum}
%%%%%%%%%%%%%%%%%%%%%%%%%%%%%%%%%%%%%%%%%%%%%%%%%%%%%%%%%%%%%%%%%%%%%%


% Mathe-Zeugs
%%%%%%%%%%%%%%%%%%%%%%%%%%%%%%%%%%%%%%%%%%%%%%%%%%%%%%%%%%%%%%%%%%%%%%
\usepackage{mathtools}

\newcommand{\BB}{\mathbb{B}}
\newcommand{\CC}{\mathbb{C}}
\newcommand{\LL}{\mathbb{L}}
\newcommand{\NN}{\mathbb{N}}
\newcommand{\QQ}{\mathbb{Q}}
\newcommand{\RR}{\mathbb{R}}
\newcommand{\ZZ}{\mathbb{Z}}
\newcommand{\oh}{\mathcal{O}}				% Landau-O
\newcommand{\ol}[1]{\overline{#1}}
\newcommand{\wt}[1]{\widetilde{#1}}
\newcommand{\wh}[1]{\widehat{#1}}

\DeclarePairedDelimiter{\absolut}{\lvert}{\rvert}		% Betrag
\DeclarePairedDelimiter{\ceiling}{\lceil}{\rceil}		% aufrunden
\DeclarePairedDelimiter{\Floor}{\lfloor}{\rfloor}		% aufrunden
\DeclarePairedDelimiter{\Norm}{\lVert}{\rVert}			% Norm
\DeclarePairedDelimiter{\sprod}{\langle}{\rangle}		% spitze Klammern
\DeclarePairedDelimiter{\enbrace}{(}{)}					% runde Klammern
\DeclarePairedDelimiter{\benbrace}{\lbrack}{\rbrack}	% eckige Klammern
\DeclarePairedDelimiter{\penbrace}{\{}{\}}				% geschweifte Klammern
\newcommand{\Underbrace}[2]{{\underbrace{#1}_{#2}}} 	% bessere Unterklammerungen
\newcommand{\abs}[1]{\absolut*{#1}}
\newcommand{\ceil}[1]{\ceiling*{#1}}
\newcommand{\flo}[1]{\Floor*{#1}}
\newcommand{\no}[1]{\Norm*{#1}}
\newcommand{\sk}[1]{\sprod*{#1}}
\newcommand{\enb}[1]{\enbrace*{#1}}
\newcommand{\penb}[1]{\penbrace*{#1}}
\newcommand{\benb}[1]{\benbrace*{#1}}
\newcommand{\stack}[2]{\makebox[1cm][c]{$\stackrel{#1}{#2}$}}
%%%%%%%%%%%%%%%%%%%%%%%%%%%%%%%%%%%%%%%%%%%%%%%%%%%%%%%%%%%%%%%%%%%%%%

\begin{document}
	%!TEX root = ../methoden.tex
% Author: Phil Steinhorst, p.st@wwu.de
\pagenumbering{roman}
\pagestyle{empty}
\begin{titlepage}
	\setlength{\parindent}{0pt}
	\begin{minipage}[b]{0.4\textwidth}
	\begin{flushleft}
	\includegraphics[width=4.25cm,keepaspectratio]{img/wwulogo17.pdf}\\[1cm]
	\end{flushleft}
	\end{minipage}
	\hfill
	\begin{minipage}[b]{0.5\textwidth}
	\begin{flushright}
	\sffamily\small
	WWU Münster \\
	Institut für Informatik
	\end{flushright}
	\end{minipage}

	\vspace{2.5cm}

	\begin{center}
		\sffamily\bfseries\Huge
		Methodensammlung

		\vspace{1cm}
		\normalfont\large\sffamily
		Eine Sammlung ausgewählter Methoden und Verfahren,\\
		die für das eine oder andere Projekt nützlich sein könnten

		\vspace{1cm}
		\normalsize\sffamily\bfseries
		erstellt von

		\vspace{0.3cm}
		\Large
		Phil Steinhorst \\
		\texttt{p.st@wwu.de}

		\vfill
		
		\normalfont\sffamily\small
		Münster, den \today
	\end{center}
\end{titlepage}

\hfill
\vfill
{
	\setlength{\parindent}{0pt}
	\small
	\textbf{Methodensammlung} \\
	Eine Sammlung ausgewählter Methoden und Verfahren,\\
	die für das eine oder andere Projekt nützlich sein könnten

	\vspace*{0.5cm}

	Phil Steinhorst \\
	Institut für Informatik, WWU Münster \\
	AG Effiziente Algorithmen und \textit{Algorithm Engineering} \\
	Einsteinstraße 62, 48149 Münster \\
	\texttt{p.st@wwu.de} \\
	Tel: +49 251 83-38411 \\
	%\url{https://www.uni-muenster.de/Informatik.AGVahrenhold/personen/philsteinhorst.html}
}

\cleardoublepage
\pagestyle{plain}
\setcounter{tocdepth}{1}
\tableofcontents
\cleardoublepage
\pagenumbering{arabic}
\setcounter{page}{1}
\pagestyle{headings}

	
	\chapter{Hallo}
	\section{Ich bin ein Test}
	\subsection{Ich auch}
	Ich bin eine Formel: $|x-y| < \delta \Rightarrow |f(x) - f(y)| < \frac{\varepsilon}{2}$.
	\lipsum[5]
	\[
		\frac{\int_{a}^{b} f(x)g(x) dx}{\int_{a}^{b} g(x) dx} \neq \int_{a}^{b} f(x) dx
	\]
	
	\begin{figure}[h]
		\centering
		\includegraphics[keepaspectratio,width=8cm]{img/wwulogo17.pdf}
		\caption{Ein Logo einer namhaften Universität.}
		\label{fig:wwulogo}
	\end{figure}

	Hier eine Quelle: \cite[S. 42]{sicp}

	\lipsum

	\index{Affe}
	\index{Äffchen}
	\index{Aerodynamik}
	\index{$a^2+b^2=c^2$}
	\index{3D-Brille}
	\index{\%-Werte}
	\index{Ohm, Georg Simon}
	\index{Ørsted, Hans Christian}
	\index{Oxford}
	\index{\textsc{öpnv}}
	\index{Wellen!Elektromagnetische}
	\index{Wellen!-berg}
	\index{Wellen!-tal}
	\index{Wellen!Teilchen-}
	\index{\verb+\tableofcontents+}

	\section{Ich bin ein zweiter Test}
	\subsection{Ich ebenfalls}

	Hier ein Verweis: Abbildung~\ref{fig:wwulogo}

	\lipsum
	\lipsum

	%!TEX root = ../methoden.tex
% Author: Phil Steinhorst, p.st@wwu.de
\addtocontents{toc}{\vspace{1em}}
{%
	\setstretch{1.1}
	\renewcommand{\bibfont}{\normalfont\small}
	\setlength{\biblabelsep}{0pt}
	\setlength{\bibitemsep}{0.5\baselineskip plus 0.5\baselineskip}
	\printbibliography
}

\vfill

Dieses Dokument wurde mit \LaTeXe\ gesetzt.

\addcontentsline{toc}{chapter}{Abbildungsverzeichnis}
\vspace*{-2.35cm}
\listoffigures

\printindex
\end{document}